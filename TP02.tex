\documentclass[11pt]{report}
\usepackage[utf8]{inputenc}
 \usepackage{listings}
 \usepackage{color}
 \usepackage{fancyhdr}
\pagestyle{fancy}

 \lhead{Geoffrey PERRIN \\ Océane DUBOIS}
 \rhead{MI01 - TP02 : VHDL séquentiel}
 \rfoot{}



\definecolor{dkgreen}{rgb}{0,0.6,0}
\definecolor{gray}{rgb}{0.5,0.5,0.5}
\definecolor{mauve}{rgb}{0.58,0,0.82}

\lstset{frame=tb,
  language=vhdl,
  aboveskip=3mm,
  belowskip=3mm,
  showstringspaces=false,
  columns=flexible,
  basicstyle={\small\ttfamily},
  numbers=none,
  numberstyle=\tiny\color{gray},
  keywordstyle=\color{blue},
  commentstyle=\color{dkgreen},
  stringstyle=\color{mauve},
  breaklines=true,
  breakatwhitespace=true,
  tabsize=3
}

 
%Gummi|065|=)
\title{\textbf{TP02 - VHDL séquentiel}
\author{Geoffrey PERRIN \\ Océane DUBOIS\\}
\date{}}

\begin{document}

\maketitle

\newpage


\section{Compteur de 2 bits}

Dans cet exercice on cherche à implémenter le  modèle VHDL d'un compteur synchrone à 2 bits, 
On utilisera : \begin{itemize}
	 \item  une entrée  pour le reset synchrone : PB\_1. Cette entrée est un bouton qu'on active si on le met à '1'.
	\item un signal d'entrée (un bouton) PB\_0 pôur l'horloge 
	 \item le signal LED\_10 comme signal de sortie, qui est un groupe de 2 led.
	 
	\end{itemize}

\medskip

Chaque fois que l'on appuie sur le bouton, le compteur est incrémenté de 1. Le résultat est affiché grâce aux diodes. Si la valeur du compteur dépasse 3, le compteur est remis à 0. Sinon le bouton reset (PB\_1), permet de remettre le compteur à 0.

\medskip

On pensera à se synchroniser sur le front montant de l'horloge.

\subsection{ a) Ecriture du programme en VHDL}

Voici le programme de l'exercice 1

 \begin{lstlisting}
entity exercice1 is
port(PB_1,PB_0 : IN BIT;
		LED_10 : OUT INTEGER range 0 to 3);
end exercice;

architecture Compteur of exercice1 is

begin
	process(PB_0)
	variable res : integer range 0 to 3;
	begin	
		if PB_1 = '0' then
				if PB_0'event and PB_0 = '1' then 
					res := res+1;
					if res > 3 then
						res := 0;
					end if;
				end if;
		else
			res := 0;
		end if;
			LED_10 <= res;
	end process;
end Compteur;

 \end{lstlisting}
 
 \subsection{ b) Synthèse du code VHDL }
 
  \subsection{ c) Simulation du circuit}
  
   \subsection{ d) programmation du FPGA}
   
   
 

\section{Détecteur de code}

Dans cet exercice on réalisera un détécteur pour le code 11010. Si le code est détecté, une alarme est activée. On utilisera les signaux :
\begin{itemize}
	\item BP\_0 pour l'horloge
	\item SW\_0 pour la ligne de transmission
	\item LED\_0 pour symboliser l'alarme
	\item LED\_7654 pour afficher l'état

\end{itemize}

Dans cette première modélisation on supposera qu'une fausse entrée implique de tout recommencer. 

\subsection{Ecriture du code VHDL}

\begin{lstlisting}
entity exercice2 is
	PORT(BP_0, SW_0 : IN BIT;
		LED_0 : OUT BIT);
end exercice2;

architecture detecteur of exercice2 is
begin
process(BP_0)
variable state : INTEGER RANGE 0 to 5;
begin
	if(PB_0'Event and PB_0 = '1') then
		case state is
			when 0 => if (SW_0 = '1') then 
						state := 1; 
						LED_0 <= '0'; 
					  end if;
			when 1 => if (SW_0 = '1') then 
						state := 2; 
					  else LED_0 <= '1'; 
						state :=0; 
					  end if;
			when 2 => if (SW_0 = '0') then 
						state := 3; 
					  else 
					  	LED_0 <= '1';
					  	 state :=0; 
					  end if;
			when 3 => if (SW_0 = '1') then 
						state := 4; 
					  else
					  	 LED_0 <= '1'; 
					  	 state :=0; 
					  end if;
			when 4 => if (SW_0 = '0') then 
						state := 5; 
					else LED_0 <= '1'; 
						state :=0;
					end if;
			when others => state := 0;
		end case;
	LED_7654 <= state;
	end if;

end process;

end detecteur;

\end{lstlisting}


\subsection{ b) Synthèse du code VHDL }
 
  \subsection{ c) Simulation du circuit}
  
  Nombre de bits utilisés par le detecteur et sous quelle forme
  Equation de génération de la sortie
  
  
   \subsection{ d) programmation du FPGA}


\section{Détecteur de code avec une fausse entrée négligée}

Dans cet exercice on réalisera un détécteur pour le code 11010. Si le code est détecté, une alarme est activée. On utilisera les signaux :
\begin{itemize}
	\item BP\_0 pour l'horloge
	\item SW\_0 pour la ligne de transmission
	\item LED\_0 pour symboliser l'alarme
	\item LED\_7654 pour afficher l'état

\end{itemize}

Dans cette première modélisation on supposera qu'une fausse entrée est négligée. 

\subsection{Ecriture du code VHDL}

\begin{lstlisting}
entity exercice3 is
	PORT(BP_0, SW_0 : IN BIT;
		LED_0 : OUT BIT);
end exercice3;

architecture detecteur of exercice3 is
begin
process(BP_0)
variable state : INTEGER RANGE 0 to 5;
begin
	if(PB_0'Event and PB_0 = '1') then
		case state is
			when 0 => if (SW_0 = '1') then 
						state := 1; 
						LED_0 <= '0'; 
					  end if;
			when 1 => if (SW_0 = '1') then 
						state := 2; 
					  else 
					  	LED_0 <= '1'; 
						state :=1; 
					  end if;
			when 2 => if (SW_0 = '0') then 
						state := 3; 
					  else 
					  	LED_0 <= '1';
					  	 state :=2; 
					  end if;
			when 3 => if (SW_0 = '1') then 
						state := 3; 
					  else
					  	 LED_0 <= '1'; 
					  	 state :=3; 
					  end if;
			when 4 => if (SW_0 = '0') then 
						state := 5; 
					else LED_0 <= '1'; 
						state :=4;
					end if;
			when others => state := 0;
		end case;
	LED_7654 <= state;
	end if;

end process;

end detecteur;

\end{lstlisting}


\subsection{ b) Synthèse du code VHDL }
 
  \subsection{ c) Simulation du circuit}
  
  Nombre de bits utilisés par le detecteur et sous quelle forme
  Equation de génération de la sortie
  
  
   \subsection{ d) programmation du FPGA}


\end{document}





























